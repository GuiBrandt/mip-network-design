\documentclass{article}
\usepackage{graphicx} % Required for inserting images
\usepackage[brazilian]{babel}
\usepackage{amsmath}
\usepackage{amssymb}
\usepackage{amsthm}
\usepackage{mathrsfs}
\usepackage{hyperref}
\usepackage{multirow}

% \usepackage[ruled,portuguese]{algorithm2e}

\title{PLI - Relatório 2: Formulação por cortes}
\author{Guilherme Guidotti Brandt (235970)}
\date{}

\newtheorem*{prop}{Proposição}
\newtheorem*{cor}{Corolário}

\begin{document}

\maketitle

\section{Formulação}

Considerando a versão direcionada do grafo (adicionando um arco em cada sentido para cada aresta no grafo original), utilizei a seguinte formulação do problema:
\begin{align}
    (\mathcal{P})\quad
    &\min \sum_{(u, v) \in A} \gamma c_{uv} x_{uv} + \sum_{(u, v) \in A} c_{uv} y_{uv}\nonumber\\
    &\sum_{v \in V} r_{v} \geq 3
        \label{min-partitions}\\
    &\sum_{v \in \delta^-(u)} y_{vu} = 1 - r_u
        &\forall u \in V\label{star-in-degree}\\
    &\sum_{v \in \delta^+(u)} d_v y_{uv} \leq r_u (B - d_u)
        &\forall u \in V\label{partition-packing}\\
    &\sum_{v \in \delta^+(u)} x_{uv} = r_u
        &\forall u \in V\label{circuit-out-degree}\\
    &\sum_{v \in \delta^-(u)} x_{vu} = r_u
        &\forall u \in V\label{circuit-in-degree}\\
    &\sum_{(u, v) \in \delta^+(S)} x_{uv} + y_{uv} + y_{vu} \geq 1
        &\forall S \subsetneq V, S \neq \emptyset\label{connectivity}\\
    &x_{uv} \in \{0, 1\}&\forall (u, v) \in A\nonumber\\
    &y_{uv} \in \{0, 1\}&\forall (u, v) \in A\nonumber\\
    &r_{v} \in \{0, 1\}&\forall v \in V\nonumber
\end{align}

Onde $A$ é o conjunto dos arcos do grafo orientado, $V$ é o conjunto de vértices no grafo, e as constantes $c_{uv} = c_{vu}$ na função de custo correspondem ao custo $c_e$ da aresta $e$ que conecta $u$ e $v$ no grafo original.

A formulação é, na maior parte, idêntica à formulação compacta apresentada na primeira parte do projeto. A única diferença entre as duas é que esta formulação não tem as variáveis $t_u$, e a restrição de eliminação de subciclos de Miller-Tucker-Zemlin é substituída pela restrição \ref{connectivity}, de conexidade, análoga à descrita por Dantzig, Fulkerson e Johnson para o problema do caixeiro viajante.

\subsection{Variáveis}

As variáveis são idênticas às da formulação compacta, exceto que, nessa formulação, não temos variáveis $t_u$ indicando a ordem dos nós no ciclo:
\begin{itemize}
    \item $x_{uv}$ indica se o arco $(u, v)$ está sendo usado no ciclo;
    \item $y_{uv}$ indica se o arco $(u, v)$ está sendo usado em alguma estrela;
    \item $r_v$ indica se o vértice $v$ está no ciclo;
\end{itemize}

\subsection{Restrições}

Assim como a formulação compacta, podemos dividir as restrições dessa formulação em três grupos: 
\begin{itemize}
    \item Restrições de partição, que garantem que a partição dos vértices respeita as condições do problema;
    \item Restrições de topologia, que garantem que a topologia da solução atende os critérios gerais do problema (exceto pela existência de subciclos);
    \item Uma restrição de conexidade, que completa as restrições de topologia, garantindo que a solução é conexa (e por consequência que não existem subciclos).
\end{itemize}

\subsubsection{Restrições de partição}

A restrição \ref{min-partitions} garante que a partição dos vértices tem pelo menos três partes (i.e. o ciclo tem tamanho maior ou igual a 3).

A restrição \ref{partition-packing} garante que a soma dos pesos dos vértices de uma estrela não ultrapasse o valor limite $B$.

\subsubsection{Restrições de topologia da solução}

A restrição \ref{star-in-degree} tem função dupla, garantindo que exatamente um arco de estrela entre em cada vértice fora do ciclo, e também que nenhum arco de estrela entra em um vértice do ciclo.

A restrição \ref{partition-packing} também garante que nenhuma aresta de estrela sai de um vértice fora do ciclo.

As restrições \ref{circuit-out-degree}  e \ref{circuit-in-degree} garantem que todo vértice no ciclo tem grau de entrada e saída (ignorando as arestas de estrela) iguais a 1, e analogamente que todo vértice fora do ciclo tem grau zero (i.e. um arco do ciclo nunca incide sobre um vértice fora do ciclo).

\subsubsection{Restrição de conexidade}

A restrição \ref{connectivity} é análoga à restrição de Dantzig-Fulkerson-Johnson para o problema do caixeiro viajante, em sua forma de cortes de aresta para grafos orientados. No nosso caso, a conexidade do grafo envolve duas variáveis: $x$ e $y$, correspondentes aos arcos do ciclo e das estrelas. Em particular, para todo subconjunto $S$ dos vértices do grafo, pelo menos um arco aresta de circuito sai de $S$, ou pelo menos um arco de estrela entra ou sai de $S$ \footnote{Mas não é possível determinar que algum dos casos dos arcos de estrela se aplique a todo corte; por exemplo, num corte que contenha apenas nós do ciclo, nenhum arco de estrela entra, e para um corte apenas com nós de estrela, nenhum arco de estrela sai.}.

A separação para essa restrição é análoga à da restrição original, e pode ser implementada de forma (relativamente) eficiente com árvore de Gomory-Hu.

\section{Experimentos computacionais}

Executei o programa em uma máquina com o sistema operacional Manjaro Linux, com 32 GB de RAM e um processador Intel i9-12900K (especificado na tabela \ref{tab:cpu-spec}), utilizando a build v10.0.3rc0 (linux64) do Gurobi, com a licença acadêmica.

\begin{table}[ht]
    \centering
    \begin{tabular}{|l|l|}
        \hline
        \textbf{Modelo} & i9-12900K \\\hline
        \textbf{Fabricante} & Intel\textsuperscript{\tiny\textregistered} \\\hline
        \textbf{Microarquitetura} & Alder Lake\\\hline
        \textbf{Núcleos} & 8 performance / 8 eficiência\\\hline
        \textbf{Threads} & 24\\\hline
        \textbf{Frequência} & 3.20GHz (performance) / 2.40 GHz (eficiência)\\\hline
        \textbf{Cache} & 30 MB Intel\textsuperscript{\tiny\textregistered} Smart Cache\\\hline
        \textbf{Cache L2} & 14 MB\\\hline
    \end{tabular}
    \caption{Especificação da CPU}
    \label{tab:cpu-spec}
\end{table}

A tabela \ref{tab:computational-results} mostra os resultados obtidos para cada instância, identificada pelo número de nós (coluna N) e o número da instância dentro do grupo de nós (coluna Inst.). A linha N = 10, Inst. = 1 corresponde à instância {\tt mo420\_network\_design\_1\_10\_100\_20\_80\_1}, a linha N = 20, Inst. = 3 corresponde à instância {\tt mo420\_network\_design\_3\_20\_100\_20\_80\_2}, etc.

A tabela \ref{tab:comparative-results} mostra os resultados obtidos comparados aos resultados da formulação compacta utilizada na parte 1.

Os tempos foram medidos utilizando a classe {\tt std::chrono::steady\_clock} da biblioteca {\tt <chrono>} da biblioteca padrão do C++ 11, com precisão de milissegundos.

\begin{table}[ht]
    \centering
    \begin{tabular}{c|c|c|c|c|c|c|c}
        \hline
        \multirow{2}{0.6cm}{N}&\multirow{2}{1cm}{Inst.} & \multicolumn{3}{c|}{Branch-And-Cut} & \multicolumn{2}{c|}{Heurística} & \multirow{2}{1.15cm}{Tempo}\\
        \cline{3-7}
        &&Valor & Lim. & Nós & Valor & Tempo \\
        \hline\hline

        \multirow{5}{0.6cm}{10}
        &{\tt 1} & 23432 & -- & 1 & 30613 & 0ms & 41ms \\
        &{\tt 2} & 29460 & -- & 1 & 40066 & 0ms & 6ms \\
        &{\tt 3} & 29315 & -- & 1 & 44758 & 0ms & 3ms \\
        &{\tt 4} & 32767 & -- & 1 & 36398 & 0ms & 19ms \\
        &{\tt 5} & 30888 & -- & 1 & 34736 & 0ms & 5ms \\

        \hline\multirow{5}{0.6cm}{20}
        &{\tt 1} & 68938 & -- & 223 & 109846 & 0ms & 219ms\\
        &{\tt 2} & 83628 & -- & 8195 & 120438 & 0ms & 1483ms \\
        &{\tt 3} & 74927 & -- & 5109 & 114112 & 0ms & 837ms \\
        &{\tt 4} & 73473 & -- & 19391 & 98290 & 0ms & 5489ms \\
        &{\tt 5} & 73429 & -- & 239 & 119636 & 0ms & 146ms\\

        \hline\multirow{2}{0.6cm}{50}
        &{\tt 1} & 247468 & 220010 & 1041381 & 336160 & 1ms & {\bf 3600s}\\
        &{\tt 2} & 257630 & 247785 & 2468317 & 367513 & 1ms & {\bf 3600s}
    \end{tabular}
    \caption{Resultados computacionais}
    \label{tab:computational-results}
\end{table}

\begin{table}[ht]
    \centering
    \begin{tabular}{c|c|c|c|c|c|c|c}
        \hline
        \multirow{2}{0.6cm}{N}&\multirow{2}{1cm}{Inst.} & \multicolumn{3}{c|}{Cortes} & \multicolumn{3}{c}{Compacta}\\
        \cline{3-8}
        &&Valor & Lim. & Tempo & Valor & Lim. & Tempo \\
        \hline\hline

        \multirow{5}{0.6cm}{10}
        &{\tt 1} & 23432 & -- & 41ms & 23432 & -- & 159ms \\
        &{\tt 2} & 29460 & -- & 6ms & 29460 & -- & 9ms \\
        &{\tt 3} & 29315 & -- & 3ms & 29315 & -- & 23ms \\
        &{\tt 4} & 32767 & -- & 19ms & 32767 & -- & 41ms \\
        &{\tt 5} & 30888 & -- & 5ms & 30888 & -- & 17ms \\

        \hline\multirow{5}{0.6cm}{20}
        &{\tt 1} & 68938 & -- & 219ms & 68938 & -- & 431ms \\
        &{\tt 2} & 83628 & -- & 1,483s & 83628 & -- & 949ms \\
        &{\tt 3} & 74927 & -- & 837ms & 74927 & -- & 339ms \\
        &{\tt 4} & 73473 & -- & 5,489s & 73473 & -- & 633ms \\
        &{\tt 5} & 73429 & -- & 146ms & 73429 & -- & 262ms \\

        \hline\multirow{2}{0.6cm}{50}
        &{\tt 1} & 247468 & 220010 & {\bf 3600s} & 244058 & -- & 163,371s\\
        &{\tt 2} & 257630 & 247785 & {\bf 3600s} & 257081 & -- & 24,106s
    \end{tabular}
    \caption{Resultados computacionais comparados à formulação compacta.}
    \label{tab:comparative-results}
\end{table}

O programa se sai bem com instâncias de até 20 nós, e apresenta tempo comparável à formulação compacta, com uma única exceção notável na instância N = 20, Inst. = 4, onde a formulação compacta dá o resultado ótimo em 633ms, enquanto a formulação por cortes leva 5,489s (quase 9 vezes mais lento).

Por outro lado, para as instâncias com 50 nós ou mais, a formulação por cortes curiosamente apresenta desempenho muito inferior ao da formulação compacta (em contraste ao TSP, onde a formulação por cortes é mais eficiente), atingindo o tempo limite em duas instâncias para as quais a formulação compacta encontra soluções ótimas em menos de três minutos (ou seja, um aumento de mais de 20 vezes no tempo).

\end{document}
